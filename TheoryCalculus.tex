\documentclass[11pt,a4paper]{article}
\usepackage[utf8]{inputenc}
\usepackage[russian]{babel}
\usepackage[OT1]{fontenc}
\usepackage{amsmath}
\usepackage{hyperref}
\usepackage{amsfonts}
\usepackage{amssymb}
\usepackage{mathtools}
\usepackage[left=2cm,right=2cm,top=2cm,bottom=2cm]{geometry}
\usepackage[italicdiff]{physics}

\hypersetup{
    colorlinks=true,
    linkcolor=blue,
    filecolor=magenta,      
    urlcolor=blue,
    pdftitle={Overleaf Example},
    pdfpagemode=FullScreen,
    }

\newcommand{\proof}{$\square$ }
\newcommand{\qed}{\hfill$\blacksquare$}
\newcommand{\pd}{\partialderivative}
\newcommand{\p}{\partial}

\newcommand{\R}{\mathbb{R}}

\setlength{\parskip}{1em}
\setlength{\parindent}{0pt}

\usepackage{fancyhdr}
\pagestyle{fancy}
\fancyhf{}
\fancyhead[C]{\textsf{ by \href{https://vk.com/ivan.dedov}{IvanUS} and \href{https://vk.com/olegsama}{OlegUS} }}

\begin{document}

\begin{center}

\begin{huge}
\textsf{Математический анализ\\1 курс}
\end{huge}

\vspace{5mm}

\begin{LARGE}
\textsf{\textbf{Теория для экзамена 4 модуля}}
\end{LARGE}

\end{center}

\textbf{1. Доказать теорему Ньютона--Лейбница. Вывести формулу Ньютона--Лейбница.\\}
\textit{Теорема Ньютона--Лейбница.} Пусть $f$ непрерывна в $(\alpha, \beta)$, $a \in (\alpha, \beta)$, $F(x) = \int_a^x f(t) dt$ -- первообразная для $f(x)$. Тогда $\forall x \in (\alpha, \beta): F'(x) = f(x)$.\\
\proof \begin{flalign*}
F'(x) & = \frac{F(x + \Delta x) - F(x)}{\Delta x} &&\\
& = \frac{1}{\Delta x} \cdot \left( \int_a^{x + \Delta x} f(t)dt - \int_a^x f(t)dt \right) &&\\
& = \frac{1}{\Delta x} \cdot \left( \int_a^x f(t)dt + \int_x^{x + \Delta x} f(t)dt - \int_a^{x} f(t)dt \right)&&\\
& = \frac{1}{\Delta x} \cdot \int_x^{x + \Delta x} f(t)dt &&\\
& f \text{ непрерывна, а значит, по теореме о среднем, } \int_x^{x + \Delta x} f(t)dt = \Delta x \cdot f(x^*), &&\\
& \text{ где } x^* \text{ лежит между } x \text{ и } x + \Delta x &&\\
& = \frac{1}{\Delta x} \cdot \Delta x \cdot f(x^*) &&\\
& = f(x^*) \xrightarrow[\Delta x \to 0]{} f(x) &&
\end{flalign*}
$F'(x) = f(x)$\qed
\\\\
\textit{Формула Ньютона--Лейбница.} Пусть $f$ -- непрерывная на $(\alpha, \beta)$, $a, b \in (\alpha, \beta)$, а $\Phi(x)$ -- некоторая первообразная для $f$. Тогда: $$\int_a^b f(x)dx = \Phi(b) - \Phi(a).$$
\proof $\exists C \in \R : \Phi (x) = F(x) + C = \int_a^x f(t)dt + C$\\
$\Phi(a) = \int_a^a f(t)dt + C = 0 + C = C$\\
$\Phi(b) = \int_a^b f(t)dt + C$\\
$\Phi(b) - \Phi(a) = \int_a^b f(t)dt + C - C = \int_a^b f(t)dt$. \qed

\textbf{2. Вывести формулу Тейлора с остаточным членом в интегральной форме.\\}
\proof
$f(x)=f(a) + f'(a)(x - a) + \frac{f''(a)}{2!}(x - a)^2 + \dots + \frac{f^{(n)}(a)}{n!} + \frac{1}{n!}\int_a^x{f^{(n+1)}(t)(x-t)^ndt}$
По формуле Ньютона-Лейбница, интегрируя по частям $n$ раз, получим:

\begin{flalign*}
f(x) - f(a) &= \int_a^x{f'(t)dt} = -\int_a^x{f'(t)d(x - t)} = -f'(t)(x - t)\Big|_a^x + \int_a^x{(x - t)f''(t)dt} =
&&\\
&=  f'(a)(x - a) - \int_a^x{f''(t)d\frac{(x - t)^2}{2}} =
&&\\
&= f'(a)(x - a) + f''(t)\frac{(x - t)^2}{2}\big|_a^x + \int_a^x{\frac{(x - t)^2}{2}df''(t)} =
&&\\
&= f'(a)(x - a) + f''(a)\frac{(x - a)^2}{2} + \int_a^x{\frac{(x - t)^2}{2}f''(t)dt} =
&&\\
&= f'(a)(x - a) + f''(a)\frac{(x - a)^2}{2} + f'''(a)\frac{(x - a)^3}{2 \cdot 3} + \int_a^x{f'''(t)\frac{(x - t)^3}{2 \cdot 3}dt} =
&&\\
&= f(a) + f'(a)(x - a) + \frac{f''(a)}{2!}(x - a)^2 + \dots + \frac{f^{(n)}(a)}{n!}(x - a)^n + \frac{1}{n!}\int_a^x{f^{(n + 1)}(t)(x - t)^ndt}
&&
\end{flalign*}
\qed

\textbf{3. Доказать признак сравнения для несобственных интегралов в предельной форме.\\}
Пусть $x \in [a, b)$, $f(x) > 0$, $g(x) > 0$. Тогда если $f(x) \sim g(x)$ при $x \rightarrow b$, то $I_1 = \int_a^b f(x)dx$ и $I_2 = \int_a^b g(x)dx$ сходятся или расходятся одновременно.\\
\proof По условию эквивалентности: $\lim_{x \to b} \frac{f(x)}{g(x)} = 1$.\\
Рассмотрим $\varepsilon = \frac{1}{2}$. Для него $\exists \delta > a : \forall x : \delta < x < b \Rightarrow \left| \frac{f(x)}{g(x)} - 1 \right| < \frac{1}{2}$\\
$\frac{1}{2} < \frac{f(x)}{g(x)} < \frac{3}{2}$\\
$\frac{1}{2} g(x) < f(x) < \frac{3}{2} g(x)$\\
Если $I_2 = \int_a^b g(x)dx$ сходится, то $\int_\delta^b f(x)dx$ сходится.\\
Если $I_2$ расходится, то $\int_\delta^b \frac{1}{2} g(x)dx$ расходится $\Rightarrow$ $\int_\delta^b f(x)dx$ расходится $\Rightarrow$ $\int_a^b f(x)dx$ расходится.\qed

\textbf{4. Доказать интегральный признак сходимости числового ряда.\\}
Дана функция $f$, определённая при всех $x \geq 1$, неотрицательная и убывающая, тогда числовой ряд $\sum_{n=1}^{\infty}{f(n)}$ сходится $\Leftrightarrow$ сходится интеграл $\int_{1}^{+\infty}{f(x)dx}$.
\\
\proof
Так как функция монотонна на $[1, +\infty)$, тогда она интегрируема по Риману на любом конечном отрезке $[1, \eta]$, и поэтому имеет смысл говорить о несобственном интеграле.
\\
Если $k \leq x \leq k + 1$, тогда $f(k) \geq f(x) \geq f(k + 1), k = 1, 2, \dots$(функция убывает). Проинтегрировав это неравенство $[k, k + 1]$, имеем:
\\
$f(k) \geq \int_{k}^{k + 1}{f(x)dx} \geq f(k + 1), k = 1, 2, \dots$.
\\
Суммируя от $k = 1$ до $k = n$, получим:
\\
$\sum_{k = 1}^{n}{f(k)} \geq \int_{1}^{n + 1}{f(x)dx} \geq \sum_{k = 1}^{n}{f(k + 1)}$
\\
Положим, $s_n = \sum_{k = 1}^{n}{f(k)}$, будем иметь
\\
$s_n \geq \int_{1}^{n + 1}{f(x)dx} \geq s_{n + 1} - f(1)$
\\
$n = 1, 2, \dots$
\\
Если интеграл сходится, то в силу неотрицательности $f$ справдливо неравенство:
\\
$\int_{1}^{n + 1}{f(x)dx} \leq \int_{1}^{+\infty}{f(x)dx}$
\\
Отсюда следует:
\\
$s_{n + 1} \leq f(1) + \int_{1}^{+\infty}{f(x)dx}$
\\
То есть последовательность частичных сумм ряда огранисена сверху, а, значит, ряд сходится.
\\
Если ряд сходится, пусть его сумма равна $s$, тогда $\forall n \in \mathbb{N}: s_n \leq s$
\\
И, следовательно, $\forall n \in \mathbb{N} \int_{1}^{n + 1}{f(x)dx}$
\\
Пусть $\xi$, то, взяв $n$ так, чтобы $n \geq \xi$
\\
В силу неотрицательности функции имеем:
\\
$\int_{1}^{\xi}{f(x)dx} \leq \int_{1}^{n}{f(x)dx} \leq s$
\\
Таким образом, совокупность всех интегралов $\int_{1}^{\xi}{f(x)dx}$ ограничена сверху.
\\
Поэтому интеграл $\int_{1}^{+\infty}{f(x)dx}$ сходится.
\qed

\textbf{5. Доказать признак д’Аламбера в предельной форме.\\}
Пусть $\lim_{n \to \infty} \frac{a_{n+1}}{a_n} = q$. Тогда:\\
$\bullet$ если $q < 1$, то ряд сходится;\\
$\bullet$ если $q > 1$, то ряд расходится;\\
$\bullet$ если $q = 1$, то имеет место неопределённость.\\
\proof 1) $\lim_{n \to \infty} \frac{a_{n+1}}{a_n} = q < 1$\\
Тогда $\exists N : \forall n \geq N \Rightarrow \frac{a_{n+1}}{a_n} < \frac{q+1}{2} < 1 \Rightarrow$ ряд сходится.\\
2) $\lim_{n \to \infty} \frac{a_{n+1}}{a_n} = q > 1$\\
Тогда $\exists N : \forall n \geq N \Rightarrow \frac{a_{n+1}}{a_n} > \frac{q+1}{2} > 1 \Rightarrow$ ряд расходится.\\
3) а) Гармонический ряд $a_n = \frac{1}{n}$; $\frac{a_{n+1}}{a_n} = \frac{\frac{1}{n+1}}{\frac{1}{n}} \xrightarrow[n \to \infty]{} 1$. Ряд расходится.\\
б) $a_n = \frac{1}{n^2}$; $\frac{a_{n+1}}{a_n} = \frac{\frac{1}{(n+1)^2}}{\frac{1}{n^2}} \xrightarrow[n \to \infty]{} 1$. Ряд сходится.\qed

\textbf{6. Доказать сходимость абсолютно сходящегося ряда.\\}
Если ряд абсолютно сходится, то он сходится.
\\
\proof
\\
$a_n^+
=
\frac{|a_n| + a_n}{2}
=
\begin{cases}
    a_n, \ if \ a_n \geq 0
    \\
    0, \ if \ a_n < 0
\end{cases}
\ \ \Rightarrow 0 \leq a_n^+ \leq |a_n| \xRightarrow[\text{по теореме сравнения}]{} \text{ряд} \ \sum_{n = 1}^{\infty}{a_n^+} \ - \ \text{сходится}$.
\\
$a_n^- = \frac{|a_n| - a_n}{2}
=
\begin{cases}
    0, \ if \ a_n \geq 0
    \\
    -a_n, \ if \ a_n < 0
\end{cases}
\Rightarrow 0 \leq a_n^- \leq |a_n| \xRightarrow[\text{по теореме сравнения}]{} \text{ряд} \ \sum_{n = 1}^{\infty}{a_n^-} \ - \ \text{сходится}$.
$\sum_{n = 1}^{\infty}{a_n} = \sum_{n = 1}^{\infty}{(a_n^+ - a_n^-)} = \underbrace{\sum_{n = 1}^{\infty}{a_n^+}}_{\text{сходится}} - \underbrace{\sum_{n = 1}^{\infty}{a_n^-}}_{\text{сходится}} \ \Rightarrow \ \text{по линейности сходится и наш ряд} \ \sum_{n = 1}^{\infty}{a_n}$.
\qed

\textbf{7. Доказать теорему о почленном интегрировании и дифференцировании функционального ряда.\\}
\textit{Теорема о почленном интегрировании функционального ряда.} Если функциональный ряд $\sum_{n=1}^{\infty} u_n (x)$ равномерно сходится в $\Delta$, и притом все $u_n (x)$ непрерывны в $\Delta$, то $$\forall [a, b] \subset \Delta : \int_a^b \left( \sum_{n=1}^{\infty} u_n (x) \right) dx = \sum_{n=1}^{\infty} \int_a^b u_n (x) dx.$$\\
\proof $s(x) = \sum_{n=1}^{\infty} u_n (x)$ непрерывна в $\Delta$ $\Rightarrow$ $s(x) \in \mathcal{R}\left( [a, b] \right)$, $[a, b] \subset \Delta$.
\begin{flalign*}
\int_a^b s(x)dx & = \int_a^b \left( s_n (x) + r_n (x) \right) dx &&\\
& = \int_a^b s_n (x) dx + \int_a^b r_n (x) dx &&\\
& = \sum_{k=1}^{n} \int_a^b u_k (x) dx + \int_a^b r_n (x) dx
\end{flalign*}
\begin{flalign*}
& \left| \int_a^b r_n (x) dx \right| \leq \left| \int_a^b \left| r_n (x) \right| dx \right| \leq \left| b - a \right| \cdot \max_{[a, b]} \left| r_n (x) \right| \xrightarrow[n \to \infty]{} 0 \text{ (по равномерной сходимости)} &&\\
& \left( \sum_{k=1}^n \int_a^b u_n (x) dx + \int_a^b r_n (x) dx \right) \xrightarrow[n \to \infty]{} \sum_{k=1}^n \int_a^b u_k (x) dx.&&\hfill\blacksquare
\end{flalign*}
\\
\textit{Теорема о почленном дифференцировании функционального ряда.} Если $\sum_{n=1}^{\infty} u_n (x)$, $x \in \Delta$ -- интервал, и притом:\\
1) $u_n (x), u_n' (x)$ непрерывны в $\Delta$;\\
2) ряд $\sum_{n=1}^{\infty} u_n' (x)$ равномерно сходится в $\Delta$;\\
3) $\exists a \in \Delta : \sum_{n=1}^{\infty} u_n (a)$ сходится,\\
то: $$\left( \sum_{n=1}^{\infty} u_n (x) \right)' = \sum_{n=1}^{\infty} u_n' (x).$$
\proof $s(x) = u_1' (x) + u_2' (x) + \hdots$ -- непрерывная в $\Delta$ функция.\\
Выясним, сходится ли $S(x) = u_1 (x) + u_2 (x) + \hdots$.\\
\begin{flalign*}
\int_a^x s(t)dt & = \int_a^x \left( \sum_{n=1}^{\infty} u_n' (t) \right) dt &&\\
& = \sum_{n=1}^{\infty} \int_a^x u_n' (t) dt &&\\
& = \underbrace{\sum_{n=1}^{\infty} \left( u_n (x) - u_n (a) \right)}_{(1)} \text{ (по Ньютону--Лейбницу)} &&\\
& = \underbrace{\sum_{n=1}^{\infty} u_n (x)}_{(2)} - \underbrace{\sum_{n=1}^{\infty} u_n (a)}_{(3)} &&
\end{flalign*}
(1), (3) сходятся по условию $\Rightarrow$ (2) сходится.\\
Таким образом, $\int_a^x s(t)dt = S(x) - S(a) \xRightarrow[]{\text{Ньютона--Лейбница}} S'(x) = s(x)$.\qed

\textbf{8. Доказать лемму Абеля.\\}
$\sum_{n = 0}^{\infty}{c_n(x - a)^n} = c_0 + c_1 \cdot (x - a) + c_2 \cdot (x - a)^2 + \dots + c_n \cdot (x - a)^n + \dots$
- степенной ряд:
\\
$a \in \R$ - центр степенного ряда
\\
$c_i \in \R$ - коэффициент степенного ряда
\\
Множество сходимости степенного ряда $\neq\varnothing$
\\
а). Если степенной ряд $\sum_{n = 0}^{\infty}{c_n \cdot x^n}$ сходится в точке $x_1 \neq 0$, то степенной ряд абсолютно сходится $\forall x: |x| < |x_1|$
\\
\proof
$\sum_{n = 0}^{\infty}{c_n \cdot x_1^n}$ сходится $\xRightarrow[\text{необходимый признак}]{} |c_nx_1^n| \xrightarrow[n \to \infty]{} 0 \Rightarrow \exists M \forall n = 0, 1, \dots \Rightarrow |c_nx_1^n| \leq M$.
\\
Пусть $|x| < |x_1| \Rightarrow |c_nx^n|=|c_n \cdot x_1^n \cdot \frac{x^n}{x_1^n}| \leq M \cdot \frac{x^n}{x_1^n}$
\\
Ряд $\sum_{n = 0}^{\infty}{|c_nx^n|}$ мажорируется (ограничен сверху) сходящимся рядом $M \cdot \sum_{n = 0}^{\infty}{\left|\frac{x}{x_n}\right|^n} < \infty$
\\
То есть, $\sum_{n = 0}^{\infty}{c_nx^n}$ абсолютно сходится.
\qed
\\
б). Если степенной ряд расходится в точке $x_2$, то степенной ряд расходится.
$\forall x: |x| > |x_2|$
\\
\proof
Если $\exists x': |x'| > |x_2|$ и $\sum_{n = 0}^{\infty}{c_n(x')^n}$ сходится, то получается противоречие пункту а).
\qed

\textbf{9. Доказать достаточное условие представимости функции рядом Тейлора.\\}
Если $f(x)$ имеет в $O_h (a)$ ($h > 0$) производные всех порядков, которые ограничены в совокупности ($\exists M > 0 : \left| f^{(n)} (x) \right| \leq M, \forall x \in O_h (a), n = 0, 1, \hdots$), то $\forall x \in O_h (a) : f(x) = \sum_{n=0}^{\infty} \frac{f^{(n)} (a)}{n!} (x - a)^n$ -- ряд Тейлора функции $f(x)$ с центром в $x = a$.\\
\proof $\forall n \forall x \in O_h (a) : f(x) = \sum_{n=0}^N \frac{f^{(n)} (a)}{n!} (x - a)^n + \frac{f^{(N+1)} (\xi)}{(N+1)!} (x - a)^{N+1}$ ($\xi$ лежит между $a$ и $x$).\\
\begin{flalign*}
& \left| \frac{f^{(N+1)} (\xi)}{(N+1)!} (x - a)^{N+1} \right| \leq M \cdot \frac{\left| x - a \right|^{N+1}}{(N+1)!} \xrightarrow[N \to \infty]{} 0 \quad (x \in O_h (a) \text{ фиксирован}) &&\\
& f(x) = \sum_{n=0}^N \frac{f^{(n)} (a)}{n!} (x - a)^n + \frac{f^{(N+1)} (\xi)}{(N+1)!} (x - a)^{N+1} \xrightarrow[N \to \infty]{} \sum_{n=0}^{\infty} \frac{f^{(n)} (a)}{n!} (x - a)^n \quad (\forall x \in O_h (a)) &&\blacksquare
\end{flalign*}

\textbf{10. Доказать теорему о необходимом условии условного экстремума.\\}
$\begin{cases}
    f(x, y, z) \rightarrow extr
    \\
    g_1(x, y, z) = 0
    \\
    g_2(x, y, z) = 0
\end{cases}$
\\
Если $M(x^*, y^*, z^*)$ - точка локального условного экстремума, $f_1, g_1, g_2$ имеют непрерывные частные производные 1 порядка в окрестности точки $M$ и $\underbrace{\nabla}_{\text{градиент}} g_1(M), \nabla g_2(M)$ - л. н. з.
\\
Утверждение: $\exists \lambda_1, \lambda_2: \nabla f(M) + \lambda_1 \nabla g_1(M) + \lambda_2 \nabla g_2(M) = 0$
\\
\proof Пусть 
$\begin{vmatrix}
    \frac{\p y_1}{\p y}&\frac{\p y_1}{\p z}
    \\
    \\
    \frac{\p y_2}{\p y}&\frac{\p y_2}{\p z}
\end{vmatrix}
\neq 0$, по теореме о неявной функции: $y = y(x); z = z(x)$ в окрестности точки $M$:
$f(x, y(x), z(x))$ и $x$ - локальные экстремумы $\phi'(x^*) = 0$
\\
В окрестности точки $M$
$\begin{cases}
    g_1(x, y(x), z(x)) = 0
    \\
    g_2(x, y(x), z(x)) = 0
\end{cases}$
\\
Рассмотрим в точке $M \frac{\p f}{\p x}$
\\
\qed

\textbf{11. Доказать теорему о достаточном условии экстремума.\\}

\end{document}